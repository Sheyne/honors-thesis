\documentclass[12pt]{article}
    % General document formatting
    \usepackage{geometry}
    \geometry{
    left=1.5in,
    top=1in,
    right=1in,
    bottom=1in
    }
    
    \usepackage{apacite}
    \usepackage{pdfpages}
    \usepackage[parfill]{parskip}
    \usepackage{amsopn}
    \usepackage{todonotes}
    \usepackage{mathtools}
    \usepackage[utf8]{inputenc}
    \usepackage{multicol}
    \usepackage{amsmath,amssymb,amsfonts,amsthm}
    \usepackage{dot2texi}
    \usepackage{tikz}
    \usepackage{mdframed}
    \usepackage[section]{placeins}
    \usepackage{mathptmx}
    \usepackage{sectsty}
    \usepackage{fancyhdr}
    % \usepackage{hyperref}
    % \usepackage[figure]{hypcap}

    \makeatletter
    \renewcommand{\section}{\@startsection
    {section}%                   % the name
    {1}%                         % the level
    {\z@}%                       % the indent / 0mm
    {-\baselineskip}%            % the before skip / -3.5ex \@plus -1ex \@minus -.2ex
    {0.5\baselineskip}%          % the after skip / 2.3ex \@plus .2ex
    {\centering\large\bfseries\MakeUppercase}} % the style
            
    \usetikzlibrary{shapes,arrows}

    \DeclareMathOperator{\StringT}{StringT}
    \DeclareMathOperator{\NumberT}{NumberT}
    \DeclareMathOperator{\BooleanT}{BooleanT}
    \DeclareMathOperator{\LitT}{LitT}
    \DeclareMathOperator{\JSLit}{\textit{JSLit}}
    \DeclareMathOperator{\JSTypeof}{JSTypeof}
    \DeclareMathOperator{\RecT}{RecT_\Gamma}
    \DeclareMathOperator{\ObjT}{ObjT_\Gamma}
    \DeclareMathOperator{\ListT}{ListT_\Gamma}
    \DeclareMathOperator{\SetT}{SetT_\Gamma}
    \DeclareMathOperator{\MapT}{MapT_\Gamma}
    \DeclareMathOperator{\ObjType}{ObjType}
    \DeclareMathOperator{\UnionT}{UnionT_\Gamma}
    \DeclareMathOperator{\InterT}{InterT_\Gamma}
    \DeclareMathOperator{\LookupObjRef}{\Gamma}
    \DeclareMathOperator{\Identifier}{\textit{Identifier}}
    \DeclareMathOperator{\Number}{Number}
    \DeclareMathOperator{\Boolean}{Boolean}
    \DeclareMathOperator{\type-ref}{ref}
    \DeclareMathOperator{\Type}{{\textit{Type}_\Gamma}}
    \DeclareMathOperator{\NoSuper}{NoSuper}
    \DeclareMathOperator{\InterOrObj}{InterOrObj}
    \DeclareMathOperator{\ObjectSubtype}{ObjectSubtype_\Gamma}
    \DeclareMathOperator{\RecordSubtype}{RecordSubtype}
    \DeclareMathOperator{\Value}{\textit{Value}_{\Gamma, \Sigma}}
    \DeclareMathOperator{\PrimitiveV}{PrimitiveV}
    \DeclareMathOperator{\RecV}{RecV_{\Gamma, \Sigma}}
    \DeclareMathOperator{\ObjV}{ObjV_{\Gamma, \Sigma}}
    \DeclareMathOperator{\ListV}{ListV_{\Gamma, \Sigma}}
    \DeclareMathOperator{\SetV}{SetV_{\Gamma, \Sigma}}
    \DeclareMathOperator{\MapV}{MapV_{\Gamma, \Sigma}}
    \DeclareMathOperator{\UnionV}{UnionV_{\Gamma, \Sigma}}
    \DeclareMathOperator{\ValueType}{ValueType_{\Gamma, \Sigma}}
    \DeclareMathOperator{\textref}{ref}
    \DeclareMathOperator{\ObjFields}{ObjFields}
    \DeclareMathOperator{\ObjPairsMatch}{ObjPairsMatch}
    \DeclareMathOperator{\where}{ where }
    \DeclareMathOperator{\textif}{ if }
    \DeclareMathOperator{\suchthat}{s.t.}
    \newcommand{\ValueRef}{\textref}
    \newcommand{\ValueDeref}[1]{\Sigma(#1)}
    \newcommand{\subtype}{<:_\Gamma}

    \newcommand{\href}[2]{\textbf{#2}}
    \newcommand{\textq}[1]{\text{``#1"}}
    
    \setcounter{secnumdepth}{0}
    \linespread{2}
    \setlength{\parindent}{1cm}
\begin{document}

\title{Sinap's Type System}
\author{Sheyne Anderson}
\renewcommand{\thepage}{\roman{page}}
\let\oldtodo\todo
\renewcommand{\todo}[1]{\oldtodo[inline]{#1}}

\includepdf[fitpaper=true, pages=1]{honorsstuff.pdf}

\section{ABSTRACT}

\todo{polish the abstract. To Matt: suggestions?}

Sinap IDE is an editor for graph-based programming languages
(as distinct from text-based).
To facilitate versatility in Sinap IDE it was necessary to design 
a language (Sinap's Type System) to generically describe families of 
cyclic data structures. This language can be used to maintain 
run-time information about values that would not otherwise be present.
It is used in Sinap IDE to store attributed graphs representing computer
programs, allowing Sinap to specialize the user interface to that graph.
This thesis presents a model for Sinap's Type System. This model 
is a formal description of the behavior of the Type System library analogous
to models of programming languages such as Lua, or Standard ML.

\newpage

\tableofcontents
\todo{remove the todolist}
\listoftodos
\newpage
\setcounter{page}{1}
\renewcommand{\headrulewidth}{0pt}
\pagestyle{fancy}
\fancyhead[R]{\thepage}
\cfoot{}
\renewcommand{\thepage}{\arabic{page}}
\section{INTRODUCTION}


Graphs are ubiquitous in computer science. They are used
in various data structures to store information, in compiler 
design to describe the structure of programs, and anywhere 
in which one might need to reason about the connections 
in a complicated system. In computation theory, graphs are 
used to describe the various models of computation such as
Deterministic Finite-State Automata (DFAs), Push Down
Automata (PDAs), and Turing Machines. 

When studying these formalisms, it is helpful 
to have an editor to create instances of these various models
and execute them on various inputs. One such editor, 
JFLAP \cite{jflap:book} is used at the University of Utah 
in CS 3100: Models of Computation. JFLAP has several drawbacks
including having an unintuitive user interface (UI), 
being difficult to maintain, and having some bugs 
\cite{jflap:challenges}. We set out to create an editor 
for programming languages that could be depicted visually as 
graphs. These include languages such as DFAs, PDAs, and Turing 
Machines, but also neural network models and some general purpose
programming languages such as LabVIEW \cite{Johnson:1997:LGP:541144}.

\subsection{Sinap IDE}

Sinap IDE is a ``generic UI platform 
for editing and interpreting domain-specific graph-based languages''
\cite{sinap-ide}. It is a tool that lets a user edit and run graphs
(using ``graph'' here to mean a system of nodes connected to edges) with
different sets of rules for how the graphs should be interpreted as a
program. These different rules are referred to as ``graph-based languages"
and the programs that run them are referred to interchangeably as 
``plugins" and ``interpreters". Sinap will adapt its user interface to 
suit the plugin that is associated with the active graph. 

The main objectives of Sinap are twofold. First, we believe that one
of the major blockers for the adoption of graph-based languages is 
a lack of adequate tooling. If someone were to invent a new text-based programming
language, they would not first have to write a text editor. The problem is
that, until we created Sinap, if someone was creating a graph based programming
language, they would first have to make a graph editor. Sinap IDE 
offers a flexible editor that will work with any graph. Second, 
the tools for manipulating automata as used throughout computation 
theory are inadequate. As noted above, JFLAP can be buggy and 
difficult to use and there are not good alternatives to it. 

Sinap IDE involved building a user interface for the graph editor and
interfaces for the various testing and editing components. 
It also involved various software design practices, such as
consistent, automatic builds, continuous integration, 
unit testing, and issue tracking. The central component of the IDE is 
the Type System which I implemented and is what the remainder of this
thesis will focus on.

While editing a graph, components of the
graph will have different attributes associated with them 
that the editor should present contextually. For example, in
a graph-language describing deterministic finite-state 
automata (DFAs) edges should have a ``symbol" attribute of 
type \texttt{character} corresponding to the input by which this 
edge should be followed. 

\begin{figure}
    % \capstart
    \centering
    \includegraphics[width=.8\textwidth]{sinap-screenshot}
    \caption{Sinap IDE editing a graph.}
    \label{sinap-screenshot}  
\end{figure}

In Figure \ref{sinap-screenshot}, Sinap's UI
presents the user with nodes and edges which can be selected 
and edited in the sidebar. Allowing the sidebar and other UI components to change based
on which interpreter is loaded is the high-level purpose of Sinap's Type System. 
This document presents a model for the type system. 

The Type System consists of a pair of languages:
one for describing data and another for describing valid structures
for that data. These two languages are analogous to XML and XML Schema
\cite{Thompson:12:WXS} respectively.

\begin{enumerate}
    \item A data-language to describe graph components 
    (implemented in values.ts)
    \item A data-description language to define valid schema 
    (implemented in types.ts)
    for the data language. For additional details see \cite{sinap-types-code}.
\end{enumerate}

The data that are allowable in the sidebar can be more complicated than
simple strings and numbers. In Figure \ref{sinap-screenshot}, the ``Position" attribute
of the node has a structure similar to the following TypeScript code:

\vspace{1ex}

\linespread{1}
\begin{verbatim}
type Point = {
    x: number;
    y: number;
}
\end{verbatim}

The goal of creating Sinap's Type System is to allow plugin/interpreter
implementers to specify what 
kinds of graphs are valid for their interpreter. This means that 
the interpreter specifies the various fields on nodes and edges 
and can assume that those fields are present on all of the graph elements
that it receives. The Type System also allows different kinds of nodes and 
edges to exist in the same graph and allows for rules defining what kinds of
nodes can connect to which edges. 

In order to better understand the Type System, I have created a 
formal model of a subset of the language. This allows proofs of
its behavior and simplifies understanding of system. This model 
is non-trivial because the Type System uses some unique types 
such as type for literals, covariant collection types, and the
ability to intersect class types together. 

\section{BACKGROUND}

For many programming languages the only specification for 
semantics is the reference implementation of the interpreter or compiler. 
Python, for example, specifies valid behavior based on the behavior of the
CPython interpreter \cite{cpython}. Other languages, such as 
C and C++ have an enormous formal document specifying the semantics
of the language. 

For studying a language academically, it is useful to create a model
of a language \cite{redexbook}. Some examples of languages with 
formal models associated with all or much of their behavior are
Featherweight Java \cite{Igarashi:2001:FJM:503502.503505}, 
Lua \cite{DBLP:journals/corr/SoldevilaZSFM17}, 
and Standard ML \cite{Milner:1997:DSM:549659}. The model is helpful because
it allows formal guarantees about the behavior of well-formed
programs. A model is often easier to 
read and more concise than a full implementation of a language. 
The remainder of this thesis will present a model of Sinap's Type 
System. 

\section{TYPES}

\subsection{Notation}

Since Sinap's Type System is a library and has no concrete 
syntax, the model adopts a contrived syntax similar to that
used in mathematics. The library is implemented in TypeScript
and the conversion is straightforward as exemplified by 
the following.

\begin{tabular}{rcl}
    \verb|new Type.Primitive("string")|&=&\(\StringT\) \\
    \verb|new Type.Literal("hello")|&=&\(\LitT(\text{``hello"})\) 
\end{tabular}

\newcommand{\Kind}{\textit{Kind}}

When discussing finite lists, the symbol 
\((\Kind_1, ...)\) is equivalent to 
\((\Kind_i)\). Both symbols are also equivalent to 
\((\Kind_1, ..., \Kind_n)\)
where \(n\) is unspecified and represents an ordered list 
of \(n\) elements. \(\{\Kind_i\}\) represents 
the set of the elements of the list. These are also valid 
as argument lists to functions. 

``\(\_\)'' represents a symbol that has been omitted for brevity. 
Because much of the model uses deeply nested parentheses, 
I will sometimes put put newlines after commas, as follows:
\begin{align*}
    (a,b,c) &= \left(\begin{aligned}
        &a,b,\\&c
    \end{aligned}\right)
\end{align*}

\subsection{Type Model}
\begin{figure}
    % \capstart
    \begin{mdframed}    
    \begin{multicols}{2}
    
    \begin{verbatim}
    class Node1 {
        label: string;
        customAttribute: number;
    }
    class Node2 {
        label: string;
        otherAttribute: {
            f1: boolean,
            f2: number
        };
    }
    class Edge {
        label: string
        source: Node1;
        destination: Node2;
    }
    type Nodes = Node1 | Node2;
    type Edges = Edge;

    
    
    \end{verbatim}
    \end{multicols}
    \end{mdframed}
    \caption{TypeScript code to go with Figure \ref{types-example}.}
    \label{types-example-code}
\end{figure}    

The type system is designed to describe families of valid graphs. 
Because of this, Sinap's Type System closely resembles the type system
of a standard programming 
language: both solve the problem of describing how data is structured.
An example to motivate Sinap's Type System is given in Figure 
\ref{types-example-code}, which shows the description of a graph-language
given in TypeScript. There are two types of nodes because, 
in general, graph languages can have different kinds of nodes in them. 
For example, in a language describing artificial neural networks, 
there might be convolutional layers, fully connected layers, and 
dropout layers as node types. Sinap Types also supports having multiple
kinds of edges. 

The model for types is composed of two parts. The first is the 
specific representation of types given in Figure 
\ref{sinap-types-model}. The second part defines a subtype
relation on these types and is given in Figure \ref{subtype-definitions}.

Before diving into Figure \ref{sinap-types-model} and 
a formal representation of each of the types, here is a presentation 
of the various types with descriptions in English. 
\begin{itemize}
    \item \(\StringT\) is a type representing all strings.
    \item \(\NumberT\) is a type representing all numbers.
    \item \(\BooleanT\) is a type representing all booleans.
    \item \(\LitT(\JSLit)\) is the type for literal primitives, specifically,
    string, number, and boolean literals. 
    Examples include \(\LitT(17), \LitT(\text{``Hello"}), \text{ and } 
    \LitT(\operatorname{false})\). This type represents only its specific literal. 
    Note that \(\LitT(17)\) is a subtype of \(\NumberT\).
    \item \(\RecT((\Identifier, \Type)...)\) is a type for records. 
    The arguments come in pairs where
    \(\Identifier\) is the name of some field and \(\Type\) is the
    type of that field.
    \item \(\ListT(\Type)\) is a type for homogeneous lists. The
    parameter is the type of all the elements of the list. 
    \item \(\SetT(\Type)\) is a type for an unordered collection
    of elements of type \(\Type\).
    \item \(\MapT((\Type)_1, (\Type)_2)\) is the type of a mapping from 
    \((\Type)_1\) to \((\Type)_2\).
    \item \(\ObjT(\Identifier, \Identifier | \NoSuper, ((\Identifier, \Type), ...))\)
    describes an object type. Its name is the first argument, 
    its super-type is the second argument with \(\NoSuper\) indicating that
    the class has no super-type. The new fields 
    the object introduces is the tuple of (key, value) pairs that form the
    last argument. To give an example, the following TypeScript
    \begin{verbatim}
class A { b : string; }
    \end{verbatim} 
    code is equivalent to \(\ObjT(``\text{A}", \NoSuper, 
    ((``\text{b}", \StringT)))\). 
    \item \(\InterT(\ObjT,  ...)\) is the intersection of several 
    object types and acts like a single object type with all the 
    properties of the intersected types. 
    To aid in understanding, \verb|class A {a : string; }| intersected
    with \verb|class B { b : string; }| acts like 
    \verb|class C {a : string; b : string; }|. 
    \item \(\UnionT(\Type, ...)\) is a union of several types. 
    Values with this type act as a collection with a single element
    whose type is one of the \(\{(\Type)_i\}\). In Sinap, a drop down 
    menu that allows ``Narrow'', ``Wide'', or some number would be 
    typed as \(\UnionT(\LitT(``\text{Narrow}"), \LitT(``\text{Wide}"),
     \NumberT)\).
\end{itemize}

\(\Gamma\) is an 
argument to the subtype relation which is used by \(\ObjectSubtype\) to 
lookup supertypes by name. These informal 
descriptions of the various types are recognized formally 
in Figure \ref{sinap-types-model}.

To come back to the example in Figure \ref{types-example-code}, 
one would write 
\[
    \operatorname{Node1} = \ObjT(``\text{Node1}", \NoSuper, (
        (``\text{label}", \StringT),
        (``\text{customAttribute}", \NumberT)
        ))
\]
to define Node1.

\linespread{1}
\begin{figure}
% \capstart
\begin{mdframed}
\begin{align*}
\Type = &\StringT   &\JSLit = &\operatorname{JSString} \\
&|\NumberT                 &&| \operatorname{JSNumber} \\
&|\BooleanT                &&| \operatorname{true} \\
&|\LitT(\JSLit)            &&| \operatorname{false} \\
&|\RecT((\Identifier, \Type), ...) & \JSTypeof(\operatorname{JSString}) &= \StringT \\
&|\ListT(\Type) & \JSTypeof(\operatorname{JSNumber}) &= \NumberT \\
&|\SetT(\Type) & \JSTypeof(\operatorname{true}) &= \BooleanT \\
&|\MapT((\Type)_1, (\Type)_2) & \JSTypeof(\operatorname{false}) &= \BooleanT \\
&|\ObjT\left(\begin{aligned}
    &\Identifier_A, \Identifier_B | \NoSuper, \\
&((\Identifier_1, \Type), ...)
\end{aligned}\right) \\
&|\InterT(\ObjT, ...) \\
&|\UnionT(\Type, ...)
\end{align*}
\end{mdframed}
\caption{Formal Description of Sinap Types.}
\label{sinap-types-model}
\end{figure}

\subsection{Subtype Relations}

With types defined, it is necessary to define a subtype 
relation (\(\bullet\subtype\bullet\)). Subtypes correspond 
roughly to the idea that if \(A\subtype B\) then \(A\)
can be used where \(B\) can be used. The subtype relation,
as formally described in Figures \ref{subtype-definitions} and \ref{subtype-helpers}, 
defines: 
\begin{itemize}
    \item literals as subtypes of their respective general types.
For example \(\LitT(\text{``Hello"})\subtype\StringT\). 
    \item record subtypes structurally; that is, all the supertype's fields 
are present in the subtype and the types of these fields in the subtype
are subtypes of the corresponding fields in the supertype. 
    So \[\RecT((\text{``a"}, \LitT(\text{``Hello"})), (\text{``b"}, \NumberT))
    <: \RecT((\text{``a"}, \StringT)))\]
    \item objects nominally, meaning that the supertype must be
found somewhere in the subtype's inheritance chain.
    \begin{align*}
        \Gamma(A) &= \ObjT(\textq{A}, \NoSuper, ()) \\
        \Gamma(B) &= \ObjT(\textq{B}, A, ()) \\
        B &<: A
    \end{align*}
    \item Unions are subtypes of other unions as long as all of the types in the 
subtype are subtypes of some type in the supertype.
    \item Intersections are subtypes of other intersections if 
    all the supertype's elements are subtypes of all the elements of the suptertype.
    Additionally, intersections are subtypes of objects that they contain.
    \item Lists, Sets, and Maps are subtypes if their arguments are subtypes. 
    That is \[\ListT(\LitT(\textq{Hello})) <: \ListT(\StringT)\].
\end{itemize}

This definition of subtypes for Lists, Maps, and Sets is unsound
as illustrated by the following pseudo-TypeScript code 
(TypeScript uses a different subtype relation).

\begin{verbatim}
    List<Dogs> dogs = [];
    // Legal, (Dog <: Animal) => (List<Dog> <: List<Animal>)
    List<Animal> animals = dogs; 
    animals.add(new Cat()); 
    // invalid, dogs[0] instanceof Cat
\end{verbatim}

This behavior is the same as Java which made this design decision 
because often lists are only read and programmers often want
this behavior. When I created the model for Sinap Types,
I noticed this defect with the logic, thereby validating 
the effort spent creating the model. 

Fortunately, in Sinap IDE, multiple references to 
the same collection are not being created,
so this defect does not come up. A later release 
of Sinap Types will include a fix for this bug in 
case Sinap Types is used in a case where this assumption is not
valid.

\newcommand{\stfif}{\\&\textif}

\begin{figure}
% \capstart
\begin{mdframed}        
\begin{align*}
    (\Type)_1&\subtype(\Type)_1 \\
    \LitT(\JSLit_1)&\subtype(\Type)_1 \stfif \JSTypeof(\JSLit_1) = (\Type)_1 \\
    \UnionT((\Type)_{1,1}, ...)&\subtype\UnionT((\Type)_{2,1}, ...) 
    \stfif \forall T\in \{(\Type)_{1,i}\} \exists T' \in \{(\Type)_{2,i}\} \suchthat T\subtype T' \\
    \InterT((\Type)_{1,1}, ...)&\subtype\InterT((\Type)_{2,1}, ...) 
    \stfif \forall T\in \{(\Type)_{1,i}\} \forall T' \in \{(\Type)_{2,i}\} T\subtype T' \\
    \InterT(..., (\Type)_i, ...)&\subtype\ObjT_1 \stfif (\Type)_i\subtype\ObjT_1  \\
    (\Type)_1 &\subtype \ObjT(\Identifier_1, \_, \_) \stfif \ObjectSubtype(\Identifier_1, \ObjT_1)\\
    (\Type)_1&\subtype(\Type)_2 \stfif \RecordSubtype(\RecT_1, \RecT_2) \\
    \ListT((\Type)_1)&\subtype\ListT((\Type)_2) \stfif (\Type)_1\subtype(\Type)_2 \\
    \SetT((\Type)_1)&\subtype\SetT((\Type)_2) \stfif (\Type)_1\subtype(\Type)_2 \\
    \MapT((\Type)_{11}, (\Type)_{12})&\subtype\MapT((\Type)_{21}, (\Type)_{22}) \stfif (\Type)_{11}\subtype(\Type)_{21} \text{ and } (\Type)_{12}\subtype(\Type)_{22} 
\end{align*}
\end{mdframed}        
\caption{Definition of the subtype relation (\(\bullet\subtype\bullet\)).}
\label{subtype-definitions}
\end{figure}
\begin{figure}
% \capstart
\begin{mdframed}        
\begin{align*}
    \ObjectSubtype&(\Identifier_1, \ObjT(\Identifier_2,\_, \_)) \\&\textif 
    \; \Identifier_1 = \Identifier_2\\
    \ObjectSubtype&(\Identifier_1, \ObjT(\_,\Identifier_2, \_)) \\&\textif 
    \;\ObjectSubtype(\Identifier_1, \LookupObjRef(\Identifier_2)))
\end{align*}
\begin{align*}
    \RecordSubtype &\left(\begin{aligned}
        &\RecT((\Identifier_{1,1}, (\Type)_{1, 1}), ..., (\Identifier_{1,n}, (\Type)_{1, n})), \\
        &\RecT((\Identifier_{2,1}, (\Type)_{2, 1}), ..., (\Identifier_{2,m}, (\Type)_{2, m}))
    \end{aligned}\right) \\
    &\textif\; \{\Identifier_{2,i}\} \subset \{\Identifier_{1,i}\}
    \text{and } \Identifier_{1, i} = \Identifier_{2, j} \\&\implies (\Type)_{1, i} \subtype (\Type)_{2, j}
\end{align*}
\end{mdframed}        
    \caption{Helper meta-functions for the subtype relation.}
\label{subtype-helpers}
\end{figure}

While Sinap's Type System is implemented as a library and can have 
concrete syntaxes in several languages, our most mature implementation 
is in TypeScript. An example of how TypeScript can be converted to 
the form used for this model is given in 
Figures \ref{types-example} and \ref{types-example-code}. 

\begin{figure}
% \capstart
\begin{mdframed}
\small
\begin{align*}
    \text{Nodes} = &\UnionT\left(
        \begin{aligned}
        &\ObjT\left(
            \begin{aligned}    
                \text{``Node1"}, \NoSuper, \left(
                    \begin{aligned}
                        &(\text{``label"}, \StringT),  \\
                        &(\text{``customAttribute"}, \NumberT)
                    \end{aligned}\right)
            \end{aligned}\right),  \\
        &\ObjT\left(\begin{aligned} &\text{``Node2"}, \NoSuper,\\&\left(\begin{aligned}
            &(\text{``label"}, \StringT),  \\
            &\left(\begin{aligned}
                \text{``otherAttribute"},\RecT\left(
                \begin{aligned}
                    (&\text{``f1"}, \BooleanT),  \\
                    (&\text{``f2"}, \NumberT)
                \end{aligned}\right)
            \end{aligned}\right)
        \end{aligned}\right)\end{aligned}\right)
        \end{aligned}\right)  \\
    \text{Edges} = &\UnionT\left(\ObjT\left(
        \text{``Edge"}, \NoSuper,  
        \left(\begin{aligned}
            (&\text{``label"}, \StringT), \\
            (&\text{``source"}, \ObjT(\text{``Node1"}, \_, \_)), \\
            (&\text{``destination"}, \ObjT(\text{``Node2"}, \_, \_))              
        \end{aligned}\right)\right)\right)
\end{align*}
\normalsize
\end{mdframed}
\caption{An example of an interpreter's description of valid graphs. 
It is equivalent to Figure \ref{types-example-code}.}
\label{types-example}
\end{figure}

\section{VALUES}

Having presented a description of valid graph structures for some
interpreter, we need to define a language for describing specific 
graphs. 
Note that Values are parameterized by \(\Sigma\), a ``store" that
maps references to \(\Value\)s. This is necessary because models are 
written in terms of trees, and using a store is the normal 
approach to allow cycles. Value terminals all contain 
exactly two arguments. The first is the specific type that the 
value is storing and the second one actually holds the data.

\begin{itemize}
    \item \(\PrimitiveV(\_, \_)\) holds
    strings, numbers, and booleans and can be typed 
    generically with \(\StringT\), \(\NumberT\), or \(\BooleanT\),
    with \(\LitT(\JSLit)\) and the appropriate literal type. 

    For example, 
    \(\PrimitiveV(\Number, 17)\) and 
    \(\PrimitiveV(\LitT(17), 17)\)
    are valid, but 
    \(\PrimitiveV(\LitT(17), 16)\)
    is not.

    Originally, primitives could also hold unions, 
    but that made Sinap's Type System very difficult 
    to understand and the functionality was removed. 
    \item \(\ObjV(\_, \_)\) holds object subtypes, specifically objects and 
    intersections. The data are arranged in the same way that object
    types are structured with a list of pairs of \((\Identifier, \Value)\).
    \item \(\RecV(\_, \_)\) holds records in much the same way that 
    objects are stored.
    \item \(\UnionV(\_, \_)\) holds a union. This merely passes the 
    responsibility of storing the value to a value corresponding to
    the member of union that is currently being stored. The purpose 
    of having this value at all is to allow the type system to 
    differentiate between a type that only allows a specific kind
    of value, and a union where the value is merely one of the options.
    \item \(\ListV(\_, \_), \MapV(\_, \_), \text{ and } \SetV(\_, \_)\)
    hold lists, maps, and sets without surprises. 
\end{itemize}

I present a formal specification of
\(\Value\) which can be found in Figure \ref{value-definition}.
Each kind of Value has a type and a body which holds the value that 
goes with the type. 
A valid graph then is defined by 
\[(\ListV(\ListT(\textit{Nodes}), \_), \ListV(\ListT(\textit{Edges}), \_))\]
for some appropriate definition of \textit{Nodes} and \textit{Edges}.
To give an example, the simple graph given in Figure \ref{simplegraph} 
might be modeled as is shown in Figure \ref{values-example1}.
The values loaded into \(\Sigma\) in Figure \ref{values-example1} can probably 
better be understood in Figure \ref{values-example2}.

\newcommand{\newlinewhere}{\\&\quad\quad\where}

\begin{figure}
% \capstart
\begin{mdframed}
\begin{align*}
    \Value =& \PrimitiveV((\Type)_1, \Identifier_1) 
    \newlinewhere (\Type)_1 \subtype \UnionT(\StringT, \NumberT, \BooleanT)  \\
    &| \ObjV((\Type)_1, V=((\Identifier_1, \ValueRef_1)), ...)) \newlinewhere
    \ObjPairsMatch(\ObjFields((\Type)_1), V) \\
    &| \RecV\left(\begin{aligned}
        &\RecT(P=((\Identifier_1, (\Type)_1), ...)),\\&V=((\Identifier_1, \ValueRef_1)), ...)
    \end{aligned}\right) \newlinewhere
    \ObjPairsMatch(P, V) \\
    &| \UnionV(\UnionT(..., (\Type)_i, ...), \ValueRef_1) \newlinewhere
    \ValueType(\ValueRef_1) \subtype (\Type)_i\\
    &| \ListV(\ListT((\Type)_1), (\ValueRef_1, ...)) \newlinewhere \ValueType(\ValueRef_i) \subtype (\Type)_1 \\
    &| \SetV(\SetT((\Type)_1), \{\ValueRef_1, ...\}) \newlinewhere \ValueType(\ValueRef_i) \subtype (\Type)_1 \\
    &| \MapV(\MapT((\Type)_1, (\Type)_2), ((\ValueRef_{1,1}, \ValueRef_{2,1}), ...)) \newlinewhere \bigwedge
    \begin{aligned}
        \ValueType(\ValueRef_{1, i}) \subtype (\Type)_1 \\
        \ValueType(\ValueRef_{2, i}) \subtype (\Type)_2 
    \end{aligned}
\end{align*}
\begin{align*}
    \ObjFields&(\ObjT(\_, \Identifier_S, (P_1 = (\Identifier_1, (\Type)_1), ...))) \\&= \operatorname{concat}(\ObjFields(\LookupObjRef(\Identifier_S), (P_n))) \\
    \ObjFields&(\InterT(ObjT_1, ...)) \\&= \operatorname{concat}(\ObjFields(\ObjT_1), ...) \\
\end{align*}
\begin{align*}
    \ValueType(\ValueRef_1) &= \ValueType(\ValueDeref{\ValueRef_1}) \\
    \ValueType(\PrimitiveV((\Type)_1, \_)) &= (\Type)_1 \\
    \ValueType(\ObjV((\Type)_1, \_)) &= (\Type)_1 \\
    \ValueType(\UnionV((\Type)_1, \_)) &= (\Type)_1 \\
    \ValueType(\ListV((\Type)_1, \_)) &= (\Type)_1 \\
    \ValueType(\SetV((\Type)_1, \_)) &= (\Type)_1 \\
    \ValueType(\MapV((\Type)_1, \_)) &= (\Type)_1
\end{align*}
\begin{align*}
    \ObjPairsMatch&(((\Identifier_1, (\Type)_1),...), ((\Identifier_1, \ValueRef_1), ...)) 
    \\&\textif \forall i, \ValueType(\Value_i) \subtype (\Type)_i
\end{align*}
\end{mdframed}
\caption{Definition of Value.}
\label{value-definition}
\end{figure}

\begin{figure}
    % \capstart
    \centering
    \begin{dot2tex}[dot, scale=0.5]
    digraph {
        "Start Node" -> "End Node";
    }
    \end{dot2tex}
    \caption{A simple graph.}
    \label{simplegraph}
\end{figure}   

\newcommand{\treeDraw}[2]{#1 \left(\begin{aligned} &#2\end{aligned}\right)}
\newcommand{\treeNext}{,\\&}
\newcommand{\valRef}[1]{\ValueRef_\textit{#1}}

\begin{figure}
% \capstart
\begin{mdframed}
\begin{align*}
    \Sigma(\valRef{0O})&=\ObjV(\ObjT_0, ())\\ 
    \Sigma(\valRef{0U})&=\UnionV(\UnionT_0, \valRef{0O})\\
    \Sigma(\valRef{1S})&=\treeDraw\PrimitiveV{\StringT_1, \textq{Start Node}}\\
    \Sigma(\valRef{1O})&=\treeDraw\ObjV{\ObjT_1,\treeDraw{}{
        (\textq{label}, \valRef{1S})\treeNext
        (\textq{destination}, \valRef{3U})
        }}\\
    \Sigma(\valRef{1U})&=\treeDraw\UnionV{\UnionT_1, \valRef{1O}}\\
    \Sigma(\valRef{2S})&=\treeDraw\PrimitiveV{\StringT_2, \textq{End Node}}\\
    \Sigma(\valRef{2O})&=\treeDraw\ObjV{\ObjT_2,\treeDraw{}{
        (\textq{label}, \valRef{2S})\treeNext
        (\textq{destination}, \text{nil})
        }}\\
    \Sigma(\valRef{2U})&=\treeDraw\UnionV{\UnionT_2, \valRef{2O}}\\
    \Sigma(\valRef{3O})&=\treeDraw\ObjV{\ObjT_3,\treeDraw{}{
        (\textq{children}, \valRef{4})
        }}\\
    \Sigma(\valRef{3U})&=\treeDraw\UnionV{\UnionT_3, \valRef{3O}}\\
    \Sigma(\valRef{4})&=\treeDraw\ListV{\ListT_4, (\valRef{2U})}\\
    \textit{Nodes}_1 &= \treeDraw\ListV{\ListT_5, (\valRef{1U}, \valRef{2U})} \\
    \textit{Edges}_1 &= \treeDraw\ListV{\ListT_6, (\valRef{3U})} 
\end{align*}
\end{mdframed}
\caption{A representation of Figure \ref{simplegraph} 
in the Sinap Type System.}
\label{values-example1}
\end{figure}

\begin{figure}
    % \capstart
    \centering
    \begin{dot2tex}[dot, scale=0.5]
    digraph {
        {rank = same; "1U"; "3U"; "2U";}
        "1U" [label="1U : UnionV"];
        "2U" [label="2U : UnionV"];
        "3U" [label="3U : UnionV"];
        "1O" [label="1O : ObjV"];
        "2O" [label="2O : ObjV"];
        "3O" [label="3O : ObjV"];
        "1S" [label="1S : PrimitiveV = Start Node"];
        "2S" [label="2S : PrimitiveV = End Node"];
        "4" [label="4 : ListV"];
        "1U" -> "1O";
        "1O" -> "1S" [label="label"];
        "2U" -> "2O" [label="label"];
        "2O" -> "2S";
        "3U" -> "3O";
        "1O" -> "3U" [label="destination"];
        "2O" -> "0U" [label="destination"];
        "3O" -> "4" [label="children"];
        "4" -> "2U"
    }
    \end{dot2tex}
\caption{A diagram of the Values in Figure \ref{values-example1}.}
\label{values-example2}
\end{figure}

\section{APPLICATIONS IN SINAP IDE}

When using the Sinap Type's System to create the IDE we added 
some invariants to specialize the Type System to graphs. 
\begin{enumerate}
    \item There must be types Nodes and Edges which are unions of \(\ObjT\) subtypes. 
    \item If any Edges type specifies ``source'' or ``destination'', it must be a Nodes subtype
    and its type will be respected and added as a constraint when creating graphs.
    \item The same goes for Nodes, except that the names are ``children'' and 
    ``parents'' and the types have to be lists of edges.
\end{enumerate}

This allows programmers writing interpreters to access connectivity information, 
a critical element of creating a graph interpreter. This also allows
for easily adding constraints to the system by adding more specific types
to the ``parents'', ``children'', ``source'', and/or ``destination'' fields.

The Sinap Type System is useful for generically describing data-structures so that 
user interfaces can specialize to handle them. We implemented Sinap's Type System
as a TypeScript library that gives runtime access to the types 
of all values used. We implemented loaders to build the type information 
for this library from TypeScript source code and to build it dynamically from 
Python. This allows Sinap Plugins to be built in Python or TypeScript. 
The TypeScript implementation is especially powerful because it allows programmers 
to create an interpreter for a graph based language and have it work with 
Sinap IDE with little to no extra effort. 

\section{CONCLUSION}

Having a formal model of the Type System helps to find
defects in the system and creates a basis atop which 
further study of the model becomes possible. 
It makes the Type System more valuable as a standalone 
system because it can be used in contexts where formal
verification techniques are desired. 

The Type System is the central component of Sinap IDE. 
In the Model View Controller (MVC) framework, the Type System
is used to implement the model for the application. The fact that
the exact structure of this model is defined not internally to
the application, but flexibly by potentially third-party plugins,
is a testament to the power that the Type System provides. 

\newpage
\bibliographystyle{apacite}
\bibliography{bib}{}

\includepdf[fitpaper=true, pages=5]{honorsstuff.pdf}


\end{document}
